\documentclass{article}
%%********************************(Packages)********************************
\usepackage{graphicx}
\usepackage{amsmath}
\usepackage{matlab-prettifier}
\usepackage{algorithm,algpseudocode}
\usepackage{tocloft}
\usepackage{listing}
\usepackage{minted}
\usepackage{amsthm}
\usepackage{amssymb}
\usepackage{pgfplots}
\usepackage{array}
\usepackage{tkz-euclide}
\usepackage{tikz}
\usepackage{cancel}
\usepackage{enumitem}
\usepackage{multicol}
\usepackage{yhmath}
\usepackage{xcolor}
\usepackage[autostyle]{csquotes}
\usepackage[hidelinks]{hyperref}
\usepackage[english,bidi=basic]{babel}
\usepackage[style=numeric,sorting=none,backend=biber]{biblatex}
%%********************************(Packages)********************************

%%********************************(Fonts)***********************************
\babelprovide[import,main]{hebrew}
\babelfont[hebrew]{rm}{David CLM}
\babelfont[hebrew]{sf}{Simple CLM}
\babelfont[hebrew]{tt}{Miriam CLM}
%%********************************(Fonts)***********************************

%%********************************(Preamble)********************************
\definecolor{braude}{RGB}{42,39,101}
\addbibresource{bibliography.bib}
\setcounter{secnumdepth}{4}
\setcounter{tocdepth}{3}
\renewcommand{\thesection}{\Roman{section}} 
\renewcommand{\thesubsection}{\arabic{subsection}}
\renewcommand{\cftsubsecdotsep}{\cftnodots}
\renewcommand\cftsecfont{\large\bfseries}
\renewcommand\cftsubsecfont{\normalsize\bfseries}
\renewcommand\cftsubsubsecfont{\normalsize}
\renewcommand\qedsymbol{$\blacksquare$}
\renewcommand{\listalgorithmname}{\begin{otherlanguage}{Hebrew}רשימת האלגוריתמים\end{otherlanguage}}
\cftsetindents{section}{0em}{2em}
\cftsetindents{subsection}{0em}{2em}
\cftsetindents{subsubsection}{2em}{2em}
\setlength{\cftbeforesecskip}{1.5\baselineskip}
\setlength{\cftbeforesubsecskip}{0.25\baselineskip}
\setlength\bibitemsep{\baselineskip}
\pgfplotsset{width=10cm,compat=1.9}
\newtheorem*{theorem}{משפט}
\newtheorem*{corollary}{מסקנה}
\newtheorem*{lemma}{למה}
\newtheorem*{definition}{הגדרה}
%%********************************(Preamble)********************************

\begin{document}
\begin{center}
    \includegraphics[scale=0.15]{figs/braude.png}\\\smallskip
    \textcolor{braude}{המחלקה למתמטיקה\\\begin{otherlanguage}{English}\textit{Department of Mathematics}\end{otherlanguage}}\\\vspace{20mm}
    {\LARGE פרויקט מסכם לתואר בוגר במדעים \begin{otherlanguage}{English}(\textsc{B.Sc})\end{otherlanguage}\\במתמטיקה שימושית}\\\vspace{20mm}
    {\fontsize{30}{36}\selectfont כאוס אינטרינזי}\\\medskip
    {\Large \textit{שהם שמיר}}\\\vspace{10mm}
    \begin{otherlanguage}{English}
        {\fontsize{30}{36}\selectfont\textsc{Intrinsic Chaos}}\\\medskip
        {\Large \textit{Shoham Shamir}}\\\vspace{35mm}
    \end{otherlanguage}
    מנחה:\hfill :\begin{otherlanguage}{English}\textit{Advisor}\end{otherlanguage}\\
    פרופ״ח חגי כתריאל\hfill\begin{otherlanguage}{English}\textit{Assoc.\hspace{0.5mm}Prof. Haggai Katriel}\end{otherlanguage}\\\vfill
    \(2023\)
\end{center}
\thispagestyle{empty}
\newpage
\begin{abstract}
\noindent במסגרת עבודת גמר זו נסקור מערכת ביולוגית של אוכלוסיית מין
יחיד חד-תאית המצויה תחת תנאים מיטיבים, הכוללים מזון וללא הימצאותם
של מינים טורפים אחרים.\bigskip
\par\noindent ההנחה הטבעית של השכל הישר היא כי במצב שכזה, כמות הפריטים באוכלוסיה
תגדל עד אשר תגיע לאיזון שבין כמות הפריטים המתווספים למערכת לאלו היוצאים
ממנה. אולם, עבור טווח ערכים של פרמטרים במערכת הנחקרת נראה כי לא כך
המצב והמערכת תִתְיַצֵּג על מנעד רחב של התנהגויות.\newline
מנעד רחב זה ינוע החל מהיכחדותה של האוכלוסיה, דרך התייצבות על-פי מחזוריות
מסוימת ועד הִוצרות של התנהגות כאוטית ללא אפשרות חיזוי — אך עם אפשרות
לתובנות בתחומי הביולוגיה והרפואה.\bigskip
\par\noindent בארבעת הפרקים בעבודה זו נציג את הרקע ההיסטורי והתפתחות תחום
תורת הכאוס, נבאר את המֹודל המוצע, ונפתח שיטות להוכחת הכאוטיות ואפיון
המערכת. בסיכום נתייחס לקושי ולחשיבות שבאפיון מערכות כאלו עם יישומיות
להמשך.
\end{abstract}
\pagenumbering{Roman}\setcounter{page}{1}
\newpage
\tableofcontents
\newpage
\listoffigures
\listofalgorithms
\newpage

\pagenumbering{arabic}\setcounter{page}{1}
\section{הקדמה}
\subsection{מידול מערכות}
\subsection{רקע מתמטי}
\subsubsection{מערכות דינמיות לא-לינאריות}
\subsubsection{שיטת רונגה-קוטה}
\subsubsection{מרחב פאזה}
\paragraph{נקודות שיווי משקל}
\paragraph{לינאריזציה}
\paragraph{יציבות של נקודות שיווי משקל}
\subsubsection{מעריכי ליאפונוב ורגישות לתנאי התחלה}
\subsubsection{ביפורקציות}
\paragraph{סוגי הסתעפויות}
\subsubsection{מושכים}
\subsection{רקע ביולוגי}

\newpage
\section{מבוא ותיאור מתמטי}
\setcounter{subsection}{3}
\subsection{מודל המערכת}
\subsubsection{סימולים}
\subsubsection{משוואות אפיון השלבים}
\subsubsection{מצבי המערכת}

\newpage
\section{שיטות ותוצאות חקר המערכת}
\setcounter{subsection}{4}
\subsection{מבחן 1–0 לכאוס}
\subsubsection{פיתוח}
\subsubsection{תוצרים}
\subsubsection{תוצאות}
\subsection{הסתעפויות}
\subsubsection{יישום}
\subsubsection{דיאגרמת ביפורקציות}
\subsection{מעריכי ליאפונוב וממד}
\subsubsection{מִקְשֶׁתֶת ליאפונוב}
\paragraph{חישוב הערכים}
\paragraph{תוצאות נומריות}
\subsubsection{שיעור קפלן-ירוק}
שיעור קפלן-יורק הינה דרך לחישוב ממד המושך בצורה מקורבת באמצעות מעריכי ליאפונוב. החישוב מבוצע כמתואר בשלבים הבאים:
\begin{enumerate}
    \item סידור המעריכים בסדר לא עולה: \[\lambda_1\geq\lambda_2\geq\ldots\geq\lambda_n\]
    \item סכימת הערכים, כל עוד הסכום אי-שלילי ומציאת האינדקס- \(j\) המתאים לכך: \[\sum_{i=1}^{j}\lambda_i\geq0,\;\sum_{i=1}^{j+1}\lambda_i<0\]
    \item חישוב ממד לפי נוסחת קפלן-יורק: \[D_{KY}=j+\frac{\sum\limits_{i=1}^{j}\lambda_i}{\lvert\lambda_{j+1}\rvert}\]
\end{enumerate}
בהרצת האלגוריתם מסעיף \(7.1.1\), כאשר- \(b_{G_2}=56,h=0.001\), התקבלו הערכים:
\[\lambda_1=0.354,\lambda_2=0,\lambda_3=-7.089,\lambda_4=-95.295\]
מכאן ש- \(j=2\), כלומר- \(D_{KY}=2+\frac{0.354+0}{\lvert-7.089\rvert}=2.05\) — ממד פרקטלי, האופייני למערכות כאוטיות.

\newpage
\section{אחרית דבר}
\phantomsection
\addcontentsline{toc}{subsection}{סיכום}
\subsection*{סיכום}
\phantomsection
\addcontentsline{toc}{subsection}{מקורות}
\subsection*{מקורות}
\begin{otherlanguage}{English}
    \nocite{*}
    \printbibliography[heading=none]
\end{otherlanguage}
\phantomsection
\addcontentsline{toc}{subsection}{נספחים}
\subsection*{נספחים}

\end{document}