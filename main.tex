\documentclass{report}
%%********************************(Packages)********************************
\usepackage{graphicx}
\usepackage[dvipsnames]{xcolor}
\usepackage[bidi=basic]{babel}
\usepackage{matlab-prettifier}
\usepackage{algorithm,algpseudocode}
\usepackage[backend=biber,style=numeric,sorting=none]{biblatex}
\usepackage[hidelinks]{hyperref}
\usepackage{amsmath}
%%********************************(Packages)********************************

%%********************************(Hebrew)**********************************
\babelprovide[main,import]{hebrew}
\babelfont[hebrew]{rm}{David CLM}
\babelfont[hebrew]{sf}{Simple CLM}
\babelfont[hebrew]{tt}{Miriam CLM}
%%********************************(Hebrew)**********************************

%%********************************(Preamble)********************************
\definecolor{braude}{rgb}{0.1647,0.15294,0.396}
\addbibresource{bibliography.bib}
%%********************************(Preamble)********************************

\begin{document}
\begin{center}
    \includegraphics[scale=0.15]{figs/braude.png}\\\smallskip
    \textcolor{braude}{המחלקה למתמטיקה\\\vspace{1.5mm}\(Department\;of\;Mathematics\)}\\\vspace{20mm}
    {\large פרויקט מסכם לתואר בוגר במדעים \(\left(B.Sc\right)\) במתמטיקה שימושית}\\\vspace{23mm}
    {\LARGE כאוס אינטרינזי במערכות\\ביולוגיות של מין יחיד}\\\medskip
    {\large שהם שמיר}\\\vspace{9mm}
    {\LARGE \(Intrinsic\;Chaos\;in\;Single\;Species\)\\\(Biological\;Systems\)}\\\medskip
    {\large  \(Shoham\;Shamir\)}\\\vspace{30mm}
    מנחה:\hfill :\(Advisor\)\\
    פרופ״ח חגי כתריאל\hfill \(Assoc.\,Prof.\:\:Haggai\;Katriel\)\\\vfill
    \(2023\)
\end{center}
\thispagestyle{empty}
\newpage
\pagenumbering{Roman}\setcounter{page}{1}

\newpage

\tableofcontents
\listoffigures
\newpage
\pagenumbering{arabic}\setcounter{page}{1}
\part{הקדמה}
\chapter{מידול מערכות}
\chapter{רקע מתמטי}
\section{מערכות דינמיות לא-לינאריות}
\section{שיטת רונגה-קוטה}
\section{מרחב פאזה}
\subsection{נקודות שיווי משקל}
\subsection{לינאריזציה}
\subsection{יציבות של נקודות שיווי משקל}
\section{מעריכי ליאפונוב ורגישות לתנאי התחלה}
\section{ביפורקציות}
\subsection{סוגי הסתעפויות}
\section{מושכים}
\chapter{רקע ביולוגי}

\part{מבוא ותיאור מתמטי}
\chapter{מודל המערכת}
\section{סימולים}
\section{משוואות אפיון השלבים}
\section{מצבי המערכת}

\part{שיטות ותוצאות חקר המערכת}
\chapter{מבחן 1–0 לכאוס}
\section{פיתוח}
\section{תוצרים}
\section{תוצאות}
\chapter{הסתעפויות}
\section{יישום}
\section{דיאגרמת ביפורקציות}
\chapter{מעריכי ליאפונוב וממד}
\section{מִקְשֶׁתֶת ליאפונוב}
\subsection{חישוב הערכים}
\subsection{תוצאות נומריות}
\section{שיעור קפלן-ירוק}
שיעור קפלן-יורק הינה דרך לחישוב ממד המושך בצורה מקורבת באמצעות מעריכי ליאפונוב. החישוב מבוצע כמתואר בשלבים הבאים:
\begin{enumerate}
    \item סידור המעריכים בסדר לא עולה: \[\lambda_1\geq\lambda_2\geq\ldots\geq\lambda_n\]
    \item סכימת הערכים, כל עוד הסכום אי-שלילי ומציאת האינדקס- \(j\) המתאים לכך: \[\sum_{i=1}^{j}\lambda_i\geq0,\;\sum_{i=1}^{j+1}\lambda_i<0\]
    \item חישוב ממד לפי נוסחת קפלן-יורק: \[D_{KY}=j+\frac{\sum\limits_{i=1}^{j}\lambda_i}{\lvert\lambda_{j+1}\rvert}\]
\end{enumerate}
בהרצת האלגוריתם מסעיף \(7.1.1\), כאשר- \(b_{G_2}=56,h=0.001\), התקבלו הערכים:
\[\lambda_1=0.354,\lambda_2=0,\lambda_3=-7.089,\lambda_4=-95.295\]
מכאן ש- \(j=2\), כלומר- \(D_{KY}=2+\frac{0.354+0}{\lvert-7.089\rvert}=2.05\) — ממד פרקטלי, האופייני למערכות כאוטיות.
\part{אחרית דבר}
\addcontentsline{toc}{chapter}{סיכום}
\section*{סיכום ומסקנות}
\addcontentsline{toc}{chapter}{מקורות}
\section*{מקורות}
\nocite{*}
\begin{otherlanguage}{English}
\printbibliography[heading=none]
\end{otherlanguage}
\addcontentsline{toc}{chapter}{נספחים}
\section*{נספחים}

\end{document}